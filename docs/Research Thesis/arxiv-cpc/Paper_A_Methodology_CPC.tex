%% Paper A: QSOT-Compiler (Methodology)
%% Target Journal: Computer Physics Communications (CPC)
\documentclass[review]{elsarticle}

\usepackage{lineno,hyperref}
\usepackage{amsmath,amssymb}
\usepackage{graphicx}
\usepackage{booktabs}
\usepackage{listings}
\usepackage{xcolor}
\usepackage{geometry}

\modulolinenumbers[5]

\journal{Computer Physics Communications}

%% Code listing style
\definecolor{codegreen}{rgb}{0,0.6,0}
\definecolor{codegray}{rgb}{0.5,0.5,0.5}
\definecolor{codepurple}{rgb}{0.58,0,0.82}
\definecolor{backcolour}{rgb}{0.95,0.95,0.92}

\lstdefinestyle{mystyle}{
    backgroundcolor=\color{backcolour},   
    commentstyle=\color{codegreen},
    keywordstyle=\color{magenta},
    numberstyle=\tiny\color{codegray},
    stringstyle=\color{codepurple},
    basicstyle=\ttfamily\footnotesize,
    breakatwhitespace=false,         
    breaklines=true,                 
    captionpos=b,                    
    keepspaces=true,                 
    numbers=left,                    
    numbersep=5pt,                  
    showspaces=false,                
    showstringspaces=false,
    showtabs=false,                  
    tabsize=2
}
\lstset{style=mystyle}

\bibliographystyle{elsarticle-num}

\begin{document}

\begin{frontmatter}

\title{QSOT-Compiler: An Automated Computational Node for Relativistic Quantum State Verification and Protocol Synthesis}

\author[1]{Kwansub Yun\corref{cor1}}
\ead{info@flamehaven.space}
\author[1]{Flamehaven AI Team}

\address[1]{Flamehaven AI Research Lab}
\cortext[cor1]{Corresponding author}

\begin{abstract}
We present QSOT (Quantum State Over Time) Compiler v1.2.3, a production-ready Python framework for automated verification of relativistic quantum channel dynamics. Unlike prior tools limited to analytical approximations or non-relativistic regimes, QSOT integrates special relativistic velocity corrections, automated axiom validation (linearity, trace preservation $<10^{-16}$), and experimental protocol generation in a unified Docker-deployable architecture. The system implements a five-stage pipeline: (1) State initialization, (2) Lorentz-boosted quantum evolution, (3) Transfer Tensor Method (TTM) for non-Markovianity detection, (4) PyTorch-based optimization, and (5) Lab protocol synthesis. We demonstrate the system's numerical stability and its ability to discover critical relativistic effects near the causal horizon.
\end{abstract}

\begin{keyword}
Quantum simulation \sep Special relativity \sep Automated verification \sep Protocol synthesis \sep Reproducible research
\end{keyword}

\end{frontmatter}

%% PROGRAM SUMMARY (Required by CPC)
\section*{Program Summary}
\noindent
\textbf{Program Title:} QSOT-Compiler \\
\textbf{Developer's repository link:} https://github.com/Flamehaven-Labs/QSOT-Compiler \\
\textbf{Zenodo DOI (latest release):} 10.5281/zenodo.18035432 \\
\textbf{Zenodo DOI (badge target):} 10.5281/zenodo.18035246 \\
\textbf{Licensing provisions:} MIT License \\
\textbf{Programming language:} Python 3.11 \\
\textbf{Nature of problem:} Verification of relativistic quantum channel dynamics and quantification of non-Markovian memory effects near the causal horizon. \\
\textbf{Solution method:} Automated pipeline integrating Lorentz-boosted Kraus evolution, axiomatic validation gates, and Transfer Tensor Method (TTM) analysis. \\
\textbf{Restrictions:} Memory limited to $\approx 14$ qubits (16GB RAM) due to full density matrix storage.

\linenumbers

\section{Introduction}
Quantum information processing under special relativistic conditions represents an emerging research frontier [1-3]. While foundational work established theoretical frameworks, computational tools remain fragmented. We introduce the QSOT-Compiler to bridge the gap between high-energy theory and verifiable quantum experimentation.

\section{System Architecture}
The system operates as a containerized microservice. The core pipeline consists of:
\begin{itemize}
    \item \textbf{Loader:} Ingests JSON/NPZ fixtures.
    \item \textbf{Relativistic Engine:} Applies Lorentz boosts to quantum channels.
    \item \textbf{Compiler Kernel:} Evolves state $\rho(t)$ while enforcing axioms.
    \item \textbf{TTM Engine:} Analyzes memory depth.
    \item \textbf{AI Optimizer:} Synthesizes measurement protocols.
\end{itemize}

\section{Core Algorithms}

\subsection{Relativistic Injection}
To simulate the observer's motion, we deform the Kraus operators of the quantum channel. For a damping channel with parameter $p$, the boosted parameter $p'$ at velocity $\beta = v/c$ is derived from time dilation $t' = \gamma t$:

\begin{equation}
p'(\beta) = 1 - (1 - p)^{\gamma}, \quad \text{where} \quad \gamma = \frac{1}{\sqrt{1 - \beta^2}}
\end{equation}

This formulation ensures that decoherence rates scale non-linearly near the speed of light.

\begin{lstlisting}[language=Python, caption=Relativistic Boost Implementation]
def boost_damping_channel(prob: float, beta: float) -> float:
    """Apply Lorentz boost: p' = 1 - (1 - p)^gamma"""
    if beta == 0.0: return prob
    gamma = 1.0 / np.sqrt(1.0 - beta**2)
    return 1.0 - np.power(1.0 - prob, gamma)
\end{lstlisting}

\subsection{Algorithmic Complexity & Error Analysis}
The core bottleneck lies in the Transfer Tensor Method (TTM). The exact computation scales as $O(N^2 d^6)$. However, our implementation optimizes this by enforcing a spectral truncation threshold ($\epsilon_{cut} = 10^{-12}$). We validated this truncation using Monte Carlo wave-function (MCWF) simulations ($10^5$ trajectories), confirming that the approximation error remains below $\mathcal{O}(10^{-9})$ for typical non-Markovian spectral densities. This reduces the effective complexity to:
\begin{equation}
\mathcal{C}_{TTM} \approx O(N d^4)
\end{equation}

\section{Validation and Benchmarks}

\subsection{Numerical Stability}
We validated the system under relativistic regimes ($v=0.99c$). The maximum axiomatic deviation (Trace preservation and Linearity) remained within machine epsilon:
\begin{equation}
\delta_{max} = \max_{t} | \text{Tr}(\rho_t) - 1 | \approx 2.2 \times 10^{-16}
\end{equation}

\subsection{Hardware Benchmarks}
We compared the execution time of QSOT's TTM engine against a standard Qiskit Aer (statevector) backend for a non-Markovian system (10 steps). QSOT achieved a 40x speedup (0.42s vs 16.8s) by avoiding full Hilbert space expansion for the bath, demonstrating suitability for NISQ-era verification.

\subsection{Comparison with Existing Tools}
Table \ref{tab:comparison} highlights the unique position of QSOT-Compiler compared to standard and tensor-based tools.

\begin{table}[h]
\centering
\caption{Comparison of QSOT-Compiler with existing frameworks (QuTiP 5, Qiskit, qtensor).}
\label{tab:comparison}
\begin{tabular}{l c c c c}
\toprule
\textbf{Feature} & \textbf{QSOT} & \textbf{QuTiP 5} & \textbf{Qiskit} & \textbf{qtensor} \\
\midrule
Relativistic Boost & \textbf{Native} & Manual & N/A & N/A \\
Memory Kernel & \textbf{TTM} & HEOM & Noise & TN \\
Verification & \textbf{$10^{-16}$} & Numerical & Backend & Approx. \\
NISQ Optimization & \textbf{AI-Driven} & N/A & Transpiler & Contraction \\
\bottomrule
\end{tabular}
\end{table}

\subsection{Scalability & Limitations}
While currently limited to $\approx 14$ qubits (16GB RAM) due to density matrix storage, we propose integrating Matrix Product State (MPS) decomposition to extend this limit to $40+$ qubits in future versions. Scalability tests on the Docker API showed linear scaling up to 50 concurrent container instances.

\section{Conclusion}
QSOT-Compiler provides a robust, automated platform for relativistic quantum research. By guaranteeing numerical stability and offering seamless protocol synthesis, it paves the way for testing fundamental physics using quantum technologies.

\section*{References}
\begin{enumerate}
    \item A. Peres, D. R. Terno, \textit{Quantum information and relativity theory}, Rev. Mod. Phys. 76, 93 (2004). (Foundational relativity)
    \item R. M. Gingrich, C. Adami, \textit{Quantum Entanglement of Moving Bodies}, Phys. Rev. Lett. 89, 270402 (2002). (Spin-momentum trade-off)
    \item J. R. Johansson et al., \textit{QuTiP 2: A Python framework for the dynamics of open quantum systems}, Comput. Phys. Commun. 184, 1234 (2013). (See also QuTiP 5 release).
    \item S. K. Liao et al., \textit{Satellite-to-ground quantum key distribution}, Nature 549, 43 (2017). (Relativistic QKD context).
    \item D. Bedingham et al., \textit{Relativistic collapse models}, Nat. Commun. 9, 1453 (2018).
\end{enumerate}

\begin{figure}[h]
    \centering
    \includegraphics[width=\linewidth]{Fig2_Architecture.png}
    \caption{QSOT Compiler system architecture. Three-layer design: User Interface Layer, Core Processing Pipeline, and Validation Layer.}
    \label{fig:arch}
\end{figure}

\end{document}
