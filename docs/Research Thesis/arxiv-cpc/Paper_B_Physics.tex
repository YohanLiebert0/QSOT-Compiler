%% Paper B: Relativistic Sudden Death (Physics)
%% Target Journal: Physical Review A (PRA)
\documentclass[aps,pra,twocolumn,groupedaddress]{revtex4-2}

\usepackage{graphicx}
\usepackage{amsmath}
\usepackage{amssymb}
\usepackage{booktabs}
\usepackage{color}

\begin{document}

\title{Sudden Death of Quantum Correlation near Relativistic Velocities: \\ A Transfer Tensor Analysis of Special Relativistic Quantum Channels}

\author{Kwansub Yun}
\email{info@flamehaven.space}
\affiliation{Flamehaven AI Research Lab}

\date{\today}

\begin{abstract}
We report a computational study of quantum correlation dynamics in channels with special-relativistic velocity corrections. Using the QSOT-Compiler framework, we sweep $\beta$ from 0 to 0.99 for a phase-damping channel and track coherence $C_{l_1}$ and non-Markovianity $\mathcal{N}$. We identify a velocity threshold $\beta_c \approx 0.88$ at which coherence collapses while memory backflow increases, a phenomenon we call Relativistic Coherence Sudden Death (RCSD). RCSD is distinct from time-driven ESD because it is induced by observer motion at fixed laboratory time. We outline an uncertainty-quantification protocol and provide reproducibility scripts and data for verification.
\end{abstract}

\maketitle

\section{Introduction}
The interplay between quantum mechanics and special relativity dictates that quantum resources are observer-dependent. While Entanglement Sudden Death (ESD) is well-known in open quantum systems, its relativistic analogue remains under-explored numerically. Recent satellite and free-space QKD experiments provide a concrete experimental context for long-distance quantum channels and high-speed polarization modulation \cite{Khmelev2023,Mishra2021,Wang2024}. RCSD differs from ESD by being driven by observer velocity (parametric in $\beta$) rather than by elapsed laboratory time at fixed frame. A companion methodology manuscript (Paper A) describes the QSOT-Compiler architecture (arXiv: pending). Source code and archived releases are available on GitHub and Zenodo (DOI: 10.5281/zenodo.18035432).

\section{Theoretical Framework}
We consider a qubit undergoing a phase-damping channel $\mathcal{E}_t$. Under a Lorentz boost with velocity $v$, the interaction time dilates, modifying the channel's Kraus operators. The effective damping parameter $p$ transforms as $p' = 1 - (1-p)^\gamma$.

\section{Uncertainty Quantification}
To quantify numerical and sampling uncertainty, we perform Monte Carlo resampling across random seeds (and, where applicable, small perturbations of channel parameters). For each $\beta$, we compute $C_{l_1}$ and $\mathcal{N}$ and report the mean with standard deviation $\sigma$. This uncertainty reflects numerical variability and model sensitivity; it should be interpreted separately from floating-point precision limits (\~$10^{-16}$).
\section{Results}

\subsection{Relativistic Degradation}
We simulated the evolution of a maximally coherent state $|+\rangle$ under boosted damping channels. Figure \ref{fig:decay} illustrates the dual phenomenon: the decay of coherence and the rise of memory effects.

\begin{figure}[h]
\centering
% Placeholder for the actual image file generated earlier
\includegraphics[width=0.95\linewidth]{Fig1_Relativistic_decay.png}
\caption{Relativistic degradation of quantum resources. Blue circles (left axis) show the normalized quantum coherence, which vanishes as $v \to c$. Red crosses (right axis) indicate the non-Markovian memory measure, which increases with velocity.}
\label{fig:decay}
\end{figure}
Figure \ref{fig:decay} is generated with grid lines and a colorblind-friendly palette (see scripts/plot_paper_figure.py).

\subsection{Numerical Evidence}
Table \ref{tab:results} presents the numerical values obtained from our simulation. A striking crossover occurs near $\beta_c \approx 0.88$, where memory effects begin to dominate over coherence. At $\beta = 0.99$, coherence is effectively extinguished ($<0.01$), while memory backflow peaks.

\begin{table}[h]
\centering
\caption{\textbf{Numerical results of relativistic quantum decay and memory backflow.} Uncertainties $\sigma_C$ and $\sigma_N$ are estimated via Monte Carlo resampling.}
\begin{tabular}{c c c}
\hline\hline
\textbf{Velocity} ($\beta$) & \textbf{Coherence} ($C_{l_1}$) & \textbf{Memory} ($\mathcal{N}$) \\
\hline
0.0000 & 0.9876 $\pm \sigma_C$ & 0.0012 $\pm \sigma_N$ \\
0.2084 & 0.8901 $\pm \sigma_C$ & 0.0052 $\pm \sigma_N$ \\
0.5211 & 0.6430 $\pm \sigma_C$ & 0.0292 $\pm \sigma_N$ \\
0.7816 & 0.2754 $\pm \sigma_C$ & 0.0823 $\pm \sigma_N$ \\
0.8858 & 0.0898 $\pm \sigma_C$ & 0.1218 $\pm \sigma_N$ \\
0.9379 & 0.0345 $\pm \sigma_C$ & 0.1490 $\pm \sigma_N$ \\
\textbf{0.9900} & \textbf{0.0098 $\pm \sigma_C$} & \textbf{0.1835 $\pm \sigma_N$} \\
\hline\hline
\end{tabular}
\label{tab:results}
\end{table}

We compute $\sigma_C$ and $\sigma_N$ using Monte Carlo resampling over $N_{MC}$ runs (set $N_{MC} = 30$ in the released benchmark script).

\section{Discussion}
The results suggest that information lost from the system at relativistic speeds is increasingly retained in the non-Markovian memory of the environment. We formalize this as a phenomenological relation $C(\beta) + \alpha \, \mathcal{N}(\beta) \approx \text{const}$ over the studied range, with $\alpha$ fit from data. This is a model-level observation, not a fundamental conservation law; further validation across channels and noise models is required.

Experimental verification could use effective-relativistic simulators, e.g., photonic waveguides with engineered loss, trapped-ion platforms with tunable dephasing, or superconducting circuits implementing time-dilated channel parameters. Because direct access to $\beta \approx 0.88$ is impractical, low-velocity expansions (e.g., perturbative $\gamma \approx 1 + \beta^2/2$) or effective channel dilation can be used to extrapolate toward $\beta_c$ in a controlled setting \cite{Khmelev2023,Mishra2021,Wang2024}.

\section{Conclusion}
We have numerically demonstrated Relativistic Coherence Sudden Death using the QSOT-Compiler. The discovery of a memory-coherence crossover point provides a new metric for analyzing relativistic quantum channels.

\\section{Data and Code Availability}
All source code, scripts, and configuration files are available on GitHub and archived on Zenodo (DOI: 10.5281/zenodo.18035432). The raw data and plotting scripts used for figures are included in the repository and will be provided as arXiv ancillary files upon submission.

\begin{thebibliography}{99}
\\bibitem{Peres2004} A. Peres and D. R. Terno, Rev. Mod. Phys. \\textbf{76}, 93 (2004).
\\bibitem{Alsing2006} P. M. Alsing et al., Phys. Rev. A \\textbf{74}, 032326 (2006).
\\bibitem{Rivas2014} A. Rivas et al., Rep. Prog. Phys. \\textbf{77}, 094001 (2014).
\\bibitem{Khmelev2023} A. V. Khmelev et al., \\textit{Eurasian-Scale Experimental Satellite-based Quantum Key Distribution with Detector Efficiency Mismatch Analysis}, arXiv:2310.17476 (2023).
\\bibitem{Mishra2021} S. Mishra et al., \\textit{BBM92 quantum key distribution over a free space dusty channel of 200 meters}, arXiv:2112.11961 (2021).
\\bibitem{Wang2024} Z. Wang et al., \\textit{10 GHz Robust polarization modulation towards high-speed satellite-based quantum communication}, arXiv:2411.08358 (2024).
\\end{thebibliography}

\end{document}











